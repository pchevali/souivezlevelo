\chapter*{La route Puno – Cusco\markboth{La route Puno – Cusco}{}}
\section*{22 mai 2015}

Cinq jours de vélo pour aller de Puno au bord du lac Titicaca à Cusco.

Un petit détour pour visiter le village de Lampa. 
\begin{center} \includegraphics[width=\mywidth]{../wp-content/uploads/2015/05/P5124011-1024x768.jpg} \end{center}
\pagebreak

Très belle église, construite par les espagnols sur des fondations incas, comme beaucoup au Pérou. 
\begin{center} \includegraphics[width=\mywidth]{../wp-content/uploads/2015/05/P5124012-1024x768.jpg} \end{center}

Les catacombes d'origine inca. 
\begin{center} \includegraphics[width=\mywidth]{../wp-content/uploads/2015/05/P5124020-1024x768.jpg} \end{center}
\pagebreak

L'église contient une reproduction de la sculpture La Pietà de Michel-Ange. 
\begin{center} \includegraphics[height=0.74\textwidth]{../wp-content/uploads/2015/05/P5124022-768x1024.jpg} \end{center}

La route traverse de belles vallées et on aperçoit parfois des sommets enneigés. 
\begin{center} \includegraphics[width=\mywidth]{../wp-content/uploads/2015/05/P5134030-1024x768.jpg} \end{center}

\begin{center} \includegraphics[width=\mywidth]{../wp-content/uploads/2015/05/P5144046-1024x768.jpg} \end{center}
\begin{center} \includegraphics[width=\mywidth]{../wp-content/uploads/2015/05/P5144050-1024x768.jpg} \end{center}
\pagebreak

Le point le plus haut de la route : col de La Raya à 4300m. 
\begin{center} \includegraphics[width=\mywidth]{../wp-content/uploads/2015/05/P5144056-1024x768.jpg} \end{center}
\begin{center} \includegraphics[width=\mywidth]{../wp-content/uploads/2015/05/P5144054-1024x768.jpg} \end{center}
\pagebreak

Bivouac dans la descente. 
\begin{center} \includegraphics[width=\mywidth]{../wp-content/uploads/2015/05/P5144064-1024x768.jpg} \end{center}

Le matin je me réveille avec un troupeau de lamas et alpagas juste au dessus de moi. 
\begin{center} \includegraphics[width=\mywidth]{../wp-content/uploads/2015/05/P5154068-1024x768.jpg} \end{center}
\pagebreak

Sur la route beaucoup de petits restaurants où on peut manger un bon repas avec une soupe et un plat pour à peine 1,5€. 
\begin{center} \includegraphics[width=\mywidth]{../wp-content/uploads/2015/05/P5134037-1024x768.jpg} \end{center}
\begin{center} \includegraphics[width=\mywidth]{../wp-content/uploads/2015/05/P5134038-1024x768.jpg} \end{center}
\pagebreak

Avant d'arriver à Cusco, plusieurs sites archéologiques à visiter.

Le site de Rachqi, situé sur le chemin de l'inca, avec les restes d'un temple, d'habitations et de bâtiments de stockage de nourriture. 
\begin{center} \includegraphics[width=\mywidth]{../wp-content/uploads/2015/05/P5154086-1024x768.jpg} \end{center}
\begin{center} \includegraphics[height=0.8\textwidth]{../wp-content/uploads/2015/05/P5154076-768x1024.jpg} \end{center}

\begin{center} \includegraphics[width=\mywidth]{../wp-content/uploads/2015/05/P5154079-1024x768.jpg} \end{center}
\begin{center} \includegraphics[width=\mywidth]{../wp-content/uploads/2015/05/P5154080-1024x768.jpg} \end{center}
\pagebreak

Dans le village d'Andihuaylillas, une belle église. 
\begin{center} \includegraphics[height=0.8\textwidth]{../wp-content/uploads/2015/05/P5164096-768x1024.jpg} \end{center}

Au bord de la route une immense porte inca. 
\begin{center} \includegraphics[width=\mywidth]{../wp-content/uploads/2015/05/P5164098-1024x768.jpg} \end{center}
\pagebreak

Puis le site pré-inca de Tiquillaka. 
\begin{center} \includegraphics[width=\mywidth]{../wp-content/uploads/2015/05/P5164109-1024x768.jpg} \end{center}
\begin{center} \includegraphics[width=\mywidth]{../wp-content/uploads/2015/05/P5164105-1024x768.jpg} \end{center}
\begin{center} \includegraphics[width=\mywidth]{../wp-content/uploads/2015/05/P5164102-1024x768.jpg} \end{center}
