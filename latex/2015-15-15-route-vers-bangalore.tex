\chapter*{Route vers Bangalore\markboth{Route vers Bangalore}{}}
\section*{15 décembre 2015}

C'est parti pour les 300 derniers kilomètres en vélo jusqu'à Bangalore. 
\begin{center} \includegraphics[width=\mywidth]{../wp-content/uploads/2015/12/wpid-oi000646-1024x768.jpg} \end{center}
\begin{center} \includegraphics[width=\mywidth]{../wp-content/uploads/2015/12/wpid-oi000648-1024x768.jpg} \end{center}
\begin{center} \includegraphics[width=\mywidth]{../wp-content/uploads/2015/12/wpid-oi000649-1024x768.jpg} \end{center}
\pagebreak

Passage par Gingee, l'occasion de monter sur l'un des forts de la ville. 
\begin{center} \includegraphics[width=\mywidth]{../wp-content/uploads/2015/12/wpid-oi000656-1024x768.jpg} \end{center}
\begin{center} \includegraphics[width=\mywidth]{../wp-content/uploads/2015/12/wpid-oi000665-1024x768.jpg} \end{center}
\begin{center} \includegraphics[width=\mywidth]{../wp-content/uploads/2015/12/wpid-oi000667-1024x768.jpg} \end{center}
\begin{center} \includegraphics[width=\mywidth]{../wp-content/uploads/2015/12/wpid-oi000674-1024x768.jpg} \end{center}
\begin{center} \includegraphics[width=\mywidth]{../wp-content/uploads/2015/12/wpid-oi000675-1024x768.jpg} \end{center}

Puis un jour à Tiruvannamalai au pied de la montagne sacrée Arunachala, destination de pèlerinage. 
\begin{center} \includegraphics[width=\mywidth]{../wp-content/uploads/2015/12/wpid-oi000683-1024x768.jpg} \end{center}
\begin{center} \includegraphics[width=\mywidth]{../wp-content/uploads/2015/12/wpid-oi000685-1024x768.jpg} \end{center}

Temple Arunachaleswara : 
\begin{center} \includegraphics[width=\mywidth]{../wp-content/uploads/2015/12/wpid-oi000684-1024x768.jpg} \end{center}
\begin{center} \includegraphics[width=\mywidth]{../wp-content/uploads/2015/12/wpid-oi000687-1024x768.jpg} \end{center}
\begin{center} \includegraphics[width=\mywidth]{../wp-content/uploads/2015/12/wpid-oi000691-1024x768.jpg} \end{center}
\begin{center} \includegraphics[width=\mywidth]{../wp-content/uploads/2015/12/wpid-oi000692-1024x768.jpg} \end{center}

L'ashram du saint Ramana Maharishi, on peut entrer dans la grotte où il est resté en méditation pendant des années. 
\begin{center} \includegraphics[width=\mywidth]{../wp-content/uploads/2015/12/wpid-oi000694-1024x768.jpg} \end{center}
\pagebreak

Montée au sommet de la montagne, un feu est entretenu en permanence. 
\begin{center} \includegraphics[width=\mywidth]{../wp-content/uploads/2015/12/wpid-oi000697-1024x768.jpg} \end{center}
\begin{center} \includegraphics[width=\mywidth]{../wp-content/uploads/2015/12/wpid-oi000698-1024x768.jpg} \end{center}
\begin{center} \includegraphics[width=\mywidth]{../wp-content/uploads/2015/12/wpid-oi000699-1024x768.jpg} \end{center}

Chaque jour, des indiens en moto se mettent à côté de moi pour discuter pendant quelques centaines de mètres, parfois plusieurs kilomètres. 
\begin{center} \includegraphics[width=\mywidth]{../wp-content/uploads/2015/12/wpid-oi000712-1024x768.jpg} \end{center}
\begin{center} \includegraphics[width=\mywidth]{../wp-content/uploads/2015/12/wpid-oi000708-1024x768.jpg} \end{center}

Dernière partie avant Bangalore sur l'autoroute. 
\begin{center} \includegraphics[width=\mywidth]{../wp-content/uploads/2015/12/wpid-oi000716-1024x768.jpg} \end{center}
\begin{center} \includegraphics[width=\mywidth]{../wp-content/uploads/2015/12/wpid-oi000718-1024x768.jpg} \end{center}

Centre de Bangalore, assez propre et moderne mais circulation énorme. 
\begin{center} \includegraphics[width=\mywidth]{../wp-content/uploads/2015/12/wpid-oi000724-1024x768.jpg} \end{center}
\begin{center} \includegraphics[width=\mywidth]{../wp-content/uploads/2015/12/wpid-oi000721-1024x768.jpg} \end{center}
\begin{center} \includegraphics[width=\mywidth]{../wp-content/uploads/2015/12/wpid-oi000727-1024x768.jpg} \end{center}

Maintenant plus que 2 semaines de voyage : ça sera vers le nord et sans le vélo.
