\chapter*{L´Equateur de Cuenca à Baños\markboth{L´Equateur de Cuenca à Baños}{}}
\section*{22 juin 2015}
Je quitte le Pérou depuis Trujillo en bus vers Cuenca : 4h de bus jusqu´à Chiclayo, suivi d´un bus de nuit vers l´Equateur avec passage de la frontière au milieu de la nuit. 

 Cuenca est une jolie ville située a 2500m d´altitude dans la sierra équatorienne. Le climat est frais et le temps très instable comme je vais m´en apercevoir sur la route ensuite. 

 

\begin{center} \includegraphics[width=\mywidth]{../wp-content/uploads/2015/06/P6124839-1024x768.jpg} \end{center}

 

 

\begin{center} \includegraphics[width=\mywidth]{../wp-content/uploads/2015/06/P6124844-1024x768.jpg} \end{center}

 

 

\begin{center} \includegraphics[width=\mywidth]{../wp-content/uploads/2015/06/P6124846-1024x768.jpg} \end{center}

 

 

\begin{center} \includegraphics[width=\mywidth]{../wp-content/uploads/2015/06/P6124849-1024x768.jpg} \end{center}

 

 

\begin{center} \includegraphics[width=\mywidth]{../wp-content/uploads/2015/06/P6124852-1024x768.jpg} \end{center}

 

 Encore beaucoup de fruits dans les marchés ici. 

 

\begin{center} \includegraphics[width=\mywidth]{../wp-content/uploads/2015/06/P6134855-1024x768.jpg} \end{center}

 

 Je visite le musée archéologique. 

 

\begin{center} \includegraphics[width=\mywidth]{../wp-content/uploads/2015/06/P6134859-768x1024.jpg} \end{center}

 

 Puis une balade au parc national Cajas près de Cuenca. 

 

\begin{center} \includegraphics[width=\mywidth]{../wp-content/uploads/2015/06/P6144861-1024x768.jpg} \end{center}

 

 

\begin{center} \includegraphics[width=\mywidth]{../wp-content/uploads/2015/06/P6144868-1024x768.jpg} \end{center}

 

 

\begin{center} \includegraphics[width=\mywidth]{../wp-content/uploads/2015/06/P6144869-1024x768.jpg} \end{center}

 

 

\begin{center} \includegraphics[width=\mywidth]{../wp-content/uploads/2015/06/P6144878-1024x768.jpg} \end{center}

 

 J´avais prévu de camper une nuit dans le parc mais l´état du chemin me fait faire demi-tour et rentrer à Cuenca. 

 

\begin{center} \includegraphics[width=\mywidth]{../wp-content/uploads/2015/06/P6144872-1024x768.jpg} \end{center}

 

 Je prends ensuite la route vers le nord pour une traversée de l´Equateur par les montagnes. 

 

\begin{center} \includegraphics[width=\mywidth]{../wp-content/uploads/2015/06/P6154884-1024x768.jpg} \end{center}

 

 

\begin{center} \includegraphics[width=\mywidth]{../wp-content/uploads/2015/06/P6154885-1024x768.jpg} \end{center}

 

 La région est très agricole, je peux observer les cultures associées de maïs et haricots. 

 

\begin{center} \includegraphics[width=\mywidth]{../wp-content/uploads/2015/06/P6154881-1024x768.jpg} \end{center}

 

 Beaucoup de chiens aussi, l´un d´entre eux a essayé de manger une sacoche arrière, heureusement c´est solide. 

 Pause dans le village d´El Tambo. 

 

\begin{center} \includegraphics[width=\mywidth]{../wp-content/uploads/2015/06/P6164891-1024x768.jpg} \end{center}

 

 La route continue avec un enchainement de cols, ca monte bien ! 

 

\begin{center} \includegraphics[width=\mywidth]{../wp-content/uploads/2015/06/P6174899-1024x768.jpg} \end{center}

 

 

\begin{center} \includegraphics[width=\mywidth]{../wp-content/uploads/2015/06/P6174901-1024x768.jpg} \end{center}

 

 Bivouac au dessus de la petite ville d´Alausi. 

 

\begin{center} \includegraphics[width=\mywidth]{../wp-content/uploads/2015/06/P6184909-1024x768.jpg} \end{center}

 

 

\begin{center} \includegraphics[width=\mywidth]{../wp-content/uploads/2015/06/P6184910-1024x768.jpg} \end{center}

 

 Je m´arrete dans un village pour faire des courses, un jeune me demande d´essayer le vélo. J´hésite mais je le laire faire : l´erreur, après seulement 5m il tombe et casse la béquille du vélo ! Au moins le vélo sera un peu plus léger maintenant…

 Encore des km et du dénivelé avant de passer par Riobamba. 

 

\begin{center} \includegraphics[width=\mywidth]{../wp-content/uploads/2015/06/P6184917-1024x768.jpg} \end{center}

 

 

\begin{center} \includegraphics[width=\mywidth]{../wp-content/uploads/2015/06/P6184918-1024x768.jpg} \end{center}

 

 

\begin{center} \includegraphics[width=\mywidth]{../wp-content/uploads/2015/06/P6194924-1024x768.jpg} \end{center}

 

 La dernière journée avant Baños est sous la pluie et dans le brouillard, en plus je me trompe de route et je fais un détour de 10km, en montée bien sur ! 

 La vue est inexistante mais heureusement des panneaux placés tous les km donnent des conseils écologiques fort utiles. 

 

\begin{center} \includegraphics[width=\mywidth]{../wp-content/uploads/2015/06/P6204934-1024x768.jpg} \end{center}

 

 J´arrive enfin à Baños de Agua Santa dans les montagnes, réputée pour ses bains chauds mais surtout pour les activités sportives à proximité : rafting, canyoning, vélo, randonnées… 

 

\begin{center} \includegraphics[width=\mywidth]{../wp-content/uploads/2015/06/P6214949-1024x768.jpg} \end{center}

 

 

\begin{center} \includegraphics[width=\mywidth]{../wp-content/uploads/2015/06/P6214951-768x1024.jpg} \end{center}

 

 

\begin{center} \includegraphics[width=\mywidth]{../wp-content/uploads/2015/06/P6204941-1024x768.jpg} \end{center}

 

 

\begin{center} \includegraphics[width=\mywidth]{../wp-content/uploads/2015/06/P6204937-1024x768.jpg} \end{center}

 

 Des stands vendent la canne à sucre sous toutes ses formes. 

 

\begin{center} \includegraphics[width=\mywidth]{../wp-content/uploads/2015/06/P6214946-1024x768.jpg} \end{center}

 

 Au dessus de Baños, la Casa de l´Arbol avec parfois une belle vue. 

 

\begin{center} \includegraphics[width=\mywidth]{../wp-content/uploads/2015/06/P6214962-768x1024.jpg} \end{center}




 
 
