\chapter*{Aurangabad et Hampi\markboth{Aurangabad et Hampi}{}}
\section*{23 décembre 2015}
Une semaine pour revenir à Bangalore, j'en profite pour faire 2 étapes touristiques à Aurangabad et Hampi 

 Première partie : 22h de train entre Delhi et Aurangabad 

 

\begin{center} \includegraphics[width=\mywidth]{../wp-content/uploads/2015/12/wpid-oi000791-1024x768.jpg} \end{center}

 

 

\begin{center} \includegraphics[width=\mywidth]{../wp-content/uploads/2015/12/wpid-oi000790-1024x768.jpg} \end{center}

 

 Une nuit dans un dortoir à l'indienne : 1.4€ 

 

\begin{center} \includegraphics[width=\mywidth]{../wp-content/uploads/2015/12/wpid-oi000793-1024x768.jpg} \end{center}

 

 Ajanta à 90km d'Aurangabad : 30 grottes bouddhistes creusées dans la falaise entre le 2e siècle avant J-C et le 8e siècle 

 

\begin{center} \includegraphics[width=\mywidth]{../wp-content/uploads/2015/12/PC151302-1024x768.jpg} \end{center}

 

 

\begin{center} \includegraphics[width=\mywidth]{../wp-content/uploads/2015/12/PC151272-1024x768.jpg} \end{center}

 

 

\begin{center} \includegraphics[width=\mywidth]{../wp-content/uploads/2015/12/PC151278-1024x768.jpg} \end{center}

 

 

\begin{center} \includegraphics[width=\mywidth]{../wp-content/uploads/2015/12/PC151280-1024x768.jpg} \end{center}

 

 

\begin{center} \includegraphics[width=\mywidth]{../wp-content/uploads/2015/12/PC151297-768x1024.jpg} \end{center}

 

 

\begin{center} \includegraphics[width=\mywidth]{../wp-content/uploads/2015/12/PC151301-1024x768.jpg} \end{center}

 

 

\begin{center} \includegraphics[width=\mywidth]{../wp-content/uploads/2015/12/PC151268-1024x768.jpg} \end{center}

 

 

\begin{center} \includegraphics[width=\mywidth]{../wp-content/uploads/2015/12/PC151307-1024x768.jpg} \end{center}

 

 

\begin{center} \includegraphics[width=\mywidth]{../wp-content/uploads/2015/12/PC151320-1024x768.jpg} \end{center}

 

 

\begin{center} \includegraphics[width=\mywidth]{../wp-content/uploads/2015/12/PC151321-1024x768.jpg} \end{center}

 

 Restes de peintures 

 

\begin{center} \includegraphics[width=\mywidth]{../wp-content/uploads/2015/12/PC151269-1024x768.jpg} \end{center}

 

 Beau site naturel autour des grottes, il doit y avoir de belles cascades pendant la saison des pluies 

 

\begin{center} \includegraphics[width=\mywidth]{../wp-content/uploads/2015/12/PC151325-1024x768.jpg} \end{center}

 

 Je rencontre un indien qui me fait visiter son village juste au dessus d'Ajanta : très tranquille, pas de voiture donc pas de klaxons incessants comme partout ailleurs 

 

\begin{center} \includegraphics[width=\mywidth]{../wp-content/uploads/2015/12/PC151327-1024x768.jpg} \end{center}

 

 

\begin{center} \includegraphics[width=\mywidth]{../wp-content/uploads/2015/12/PC151328-1024x768.jpg} \end{center}

 

 

\begin{center} \includegraphics[width=\mywidth]{../wp-content/uploads/2015/12/PC151329-1024x768.jpg} \end{center}

 

 L'école du village 

 

\begin{center} \includegraphics[width=\mywidth]{../wp-content/uploads/2015/12/PC151330-1024x768.jpg} \end{center}

 

 Ellora juste à côté d'Aurangabad : un mélange de grottes hindoues, jainistes et bouddhistes 

 Temple hindou de Kailasanatha, entièrement excavé dans la falaise 

 

\begin{center} \includegraphics[width=\mywidth]{../wp-content/uploads/2015/12/PC161360-1024x768.jpg} \end{center}

 

 

\begin{center} \includegraphics[width=\mywidth]{../wp-content/uploads/2015/12/PC161356-1024x768.jpg} \end{center}

 

 Groupe de 4 grottes jainistes 

 

\begin{center} \includegraphics[width=\mywidth]{../wp-content/uploads/2015/12/PC161387-1024x768.jpg} \end{center}

 

 

\begin{center} \includegraphics[width=\mywidth]{../wp-content/uploads/2015/12/PC161390-1024x768.jpg} \end{center}

 

 

\begin{center} \includegraphics[width=\mywidth]{../wp-content/uploads/2015/12/PC161392-768x1024.jpg} \end{center}

 

 Une dizaines de grottes bouddhistes, ayant servies de temple ou de monastère, ça attire des touristes particuliers 

 

\begin{center} \includegraphics[width=\mywidth]{../wp-content/uploads/2015/12/PC161409-1024x768.jpg} \end{center}

 

 

\begin{center} \includegraphics[width=\mywidth]{../wp-content/uploads/2015/12/PC161410-1024x768.jpg} \end{center}

 

 

\begin{center} \includegraphics[width=\mywidth]{../wp-content/uploads/2015/12/PC161411-1024x768.jpg} \end{center}

 

 

\begin{center} \includegraphics[width=\mywidth]{../wp-content/uploads/2015/12/PC161416-1024x768.jpg} \end{center}

 

 

\begin{center} \includegraphics[width=\mywidth]{../wp-content/uploads/2015/12/PC161418-1024x768.jpg} \end{center}

 

 

\begin{center} \includegraphics[width=\mywidth]{../wp-content/uploads/2015/12/PC161420-1024x768.jpg} \end{center}

 

 

\begin{center} \includegraphics[width=\mywidth]{../wp-content/uploads/2015/12/PC161428-1024x768.jpg} \end{center}

 

 Je prends ensuite le bus pour rejoindre le village d'Hampi, il m'avait été conseillé par plusieurs voyageurs que j'ai rencontrés 

 

\begin{center} \includegraphics[width=\mywidth]{../wp-content/uploads/2015/12/PC181547-1024x768.jpg} \end{center}

 

 Cadre naturel très particulier : des cailloux partout 

 

\begin{center} \includegraphics[width=\mywidth]{../wp-content/uploads/2015/12/PC171439-1024x768.jpg} \end{center}

 

 

\begin{center} \includegraphics[width=\mywidth]{../wp-content/uploads/2015/12/PC181469-1024x768.jpg} \end{center}

 

 

\begin{center} \includegraphics[width=\mywidth]{../wp-content/uploads/2015/12/PC181478-1024x768.jpg} \end{center}

 

 

\begin{center} \includegraphics[width=\mywidth]{../wp-content/uploads/2015/12/PC181550-1024x768.jpg} \end{center}

 

 Monkey temple en haut d'une colline 

 

\begin{center} \includegraphics[width=\mywidth]{../wp-content/uploads/2015/12/PC171456-1024x768.jpg} \end{center}

 

 

\begin{center} \includegraphics[width=\mywidth]{../wp-content/uploads/2015/12/PC171453-1024x768.jpg} \end{center}

 

 Temple principal Virupaksha 

 

\begin{center} \includegraphics[width=\mywidth]{../wp-content/uploads/2015/12/PC171460-1024x768.jpg} \end{center}

 

 

\begin{center} \includegraphics[width=\mywidth]{../wp-content/uploads/2015/12/PC181464-1024x768.jpg} \end{center}

 

 

\begin{center} \includegraphics[width=\mywidth]{../wp-content/uploads/2015/12/PC181548-1024x768.jpg} \end{center}

 

 

\begin{center} \includegraphics[width=\mywidth]{../wp-content/uploads/2015/12/PC181541-1024x768.jpg} \end{center}

 

 Les ruines de Vijayanâgara, ancienne capitale Hindoue, éparpillées tout autour de Hampi 

 

\begin{center} \includegraphics[width=\mywidth]{../wp-content/uploads/2015/12/PC181482-1024x768.jpg} \end{center}

 

 

\begin{center} \includegraphics[width=\mywidth]{../wp-content/uploads/2015/12/PC181486-1024x768.jpg} \end{center}

 

 

\begin{center} \includegraphics[width=\mywidth]{../wp-content/uploads/2015/12/PC181487-1024x768.jpg} \end{center}

 

 

\begin{center} \includegraphics[width=\mywidth]{../wp-content/uploads/2015/12/PC181509-1024x768.jpg} \end{center}

 

 

\begin{center} \includegraphics[width=\mywidth]{../wp-content/uploads/2015/12/PC181499-1024x768.jpg} \end{center}

 

 

\begin{center} \includegraphics[width=\mywidth]{../wp-content/uploads/2015/12/PC181510-1024x768.jpg} \end{center}

 

 

\begin{center} \includegraphics[width=\mywidth]{../wp-content/uploads/2015/12/PC181521-1024x768.jpg} \end{center}

 

 

\begin{center} \includegraphics[width=\mywidth]{../wp-content/uploads/2015/12/PC181540-1024x768.jpg} \end{center}

 

 Lotus Mahal 

 

\begin{center} \includegraphics[width=\mywidth]{../wp-content/uploads/2015/12/PC181526-1024x768.jpg} \end{center}

 

 Etable pour éléphants 

 

\begin{center} \includegraphics[width=\mywidth]{../wp-content/uploads/2015/12/PC181529-1024x768.jpg} \end{center}

 

 Bains de la reine 

 

\begin{center} \includegraphics[width=\mywidth]{../wp-content/uploads/2015/12/PC181536-1024x768.jpg} \end{center}

 

 Dernière nuit de train pour rentrer à Bangalore, je récupère le vélo laissé chez Abhijit qui m'a hébergé 

 

\begin{center} \includegraphics[width=\mywidth]{../wp-content/uploads/2015/12/PC191554-1024x768.jpg} \end{center}

 

 Puis 30 km pour rejoindre l'aéroport et c'est la fin de 10 mois et demi de voyage !


 
 
