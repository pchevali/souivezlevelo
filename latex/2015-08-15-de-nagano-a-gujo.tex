\chapter{De Nagano à Gujō}
\section*{15 août 2015}
Je quitte Nagano par le sud en direction de Matsumoto. \newline
 \newline
\centerline{\includegraphics[height=90mm]{../wp-content/uploads/2015/08/P8035933-1024x768.jpg} } 
 \newline
 A la sortie de la ville, je m'arrête dans un magasin de vélo pour changer les plateaux et la chaine. Le technicien monte la chaine neuve que j'avais déjà et 2 plateaux récupérés sur un vélo dans le magasin. A la fin je demande pour payer mais il me dit simplement «Good bye it's service», incroyable ! \newline
 \newline
\centerline{\includegraphics[height=90mm]{../wp-content/uploads/2015/08/P8045944-1024x768.jpg} } 
 \newline
 A Matsumoto, visite du célèbre chateau qui est d'origine, contrairement à beaucoup de chateaux au Japon. \newline
 \newline
\centerline{\includegraphics[height=90mm]{../wp-content/uploads/2015/08/P8045947-1024x768.jpg} } 
 \newline
 \newline
\centerline{\includegraphics[height=90mm]{../wp-content/uploads/2015/08/P8045951-1024x768.jpg} } 
 \newline
 \newline
\centerline{\includegraphics[height=90mm]{../wp-content/uploads/2015/08/P8045953-1024x768.jpg} } 
 \newline
 \newline
\centerline{\includegraphics[height=90mm]{../wp-content/uploads/2015/08/P8045959-1024x768.jpg} } 
 \newline
 \newline
\centerline{\includegraphics[height=90mm]{../wp-content/uploads/2015/08/P8045961-1024x768.jpg} } 
 \newline
 Ensuite la route monte pour entrer dans les Alpes Japonaises, portion pas toujours agréable avec beaucoup de tunnels dont un de 2km en montée à plus de 12%. \newline
 \newline
\centerline{\includegraphics[height=90mm]{../wp-content/uploads/2015/08/P8045969-1024x768.jpg} } 
 \newline
 \newline
\centerline{\includegraphics[height=90mm]{../wp-content/uploads/2015/08/P8045970-1024x768.jpg} } 
 \newline
 \newline
\centerline{\includegraphics[height=90mm]{../wp-content/uploads/2015/08/P8045972-1024x768.jpg} } 
 \newline
 La montée se termine à Kamikochi par une portion de route plus tranquille ouverte seulement aux bus et aux taxis. \newline
 \newline
\centerline{\includegraphics[height=90mm]{../wp-content/uploads/2015/08/P8055978-1024x768.jpg} } 
 \newline
 \newline
\centerline{\includegraphics[height=90mm]{../wp-content/uploads/2015/08/P8055985-1024x768.jpg} } 
 \newline
 C'est le lieu de départ de plusieurs randonnées et ascensions, j'en profite pour monter jusqu'à un refuge à 2200m. \newline
 \newline
\centerline{\includegraphics[height=90mm]{../wp-content/uploads/2015/08/P8055987-1024x768.jpg} } 
 \newline
 \newline
\centerline{\includegraphics[height=90mm]{../wp-content/uploads/2015/08/P8055990-1024x768.jpg} } 
 \newline
 \newline
\centerline{\includegraphics[height=90mm]{../wp-content/uploads/2015/08/P8055993-1024x768.jpg} } 
 \newline
 \newline
\centerline{\includegraphics[height=90mm]{../wp-content/uploads/2015/08/P8055998-1024x768.jpg} } 
 \newline
 En fin d'après midi je redescends un peu et me lance dans un nouveau col quand je suis surpris par un orage. La pluie ne s'arrête pas, je me decide à poser la tente : idée moyenne, tout est trempé y compris l'intérieur qui est une grosse flaque ! \newline
 \newline
\centerline{\includegraphics[height=90mm]{../wp-content/uploads/2015/08/P8056003-1024x768.jpg} } 
 \newline
 \newline
\centerline{\includegraphics[height=90mm]{../wp-content/uploads/2015/08/P8056001-1024x768.jpg} } 
 \newline
 Après 2 autres cols, une longue descente me mène à Takayama ou je reste 2 jours dans un hotel pour me reposer et sécher complètement. \newline
 \newline
\centerline{\includegraphics[height=90mm]{../wp-content/uploads/2015/08/P8066004-1024x768.jpg} } 
 \newline
 \newline
\centerline{\includegraphics[height=90mm]{../wp-content/uploads/2015/08/P8066005-1024x768.jpg} } 
 \newline
 \newline
\centerline{\includegraphics[height=90mm]{../wp-content/uploads/2015/08/P8066012-1024x768.jpg} } 
 \newline
 Un petit marché se tient au bord de la rivière. \newline
 \newline
\centerline{\includegraphics[height=90mm]{../wp-content/uploads/2015/08/P8076023-e1439361015244-768x1024.jpg} } 
 \newline
 \newline
\centerline{\includegraphics[height=90mm]{../wp-content/uploads/2015/08/P8076021-1024x768.jpg} } 
 \newline
 Beaucoup de maisons traditionnelles en bois dans le centre ville. \newline
 \newline
\centerline{\includegraphics[height=90mm]{../wp-content/uploads/2015/08/P8076026-1024x768.jpg} } 
 \newline
 \newline
\centerline{\includegraphics[height=90mm]{../wp-content/uploads/2015/08/P8076030-1024x768.jpg} } 
 \newline
 \newline
\centerline{\includegraphics[height=90mm]{../wp-content/uploads/2015/08/P8076034-1024x768.jpg} } 
 \newline
 \newline
\centerline{\includegraphics[height=90mm]{../wp-content/uploads/2015/08/P8076041-1024x768.jpg} } 
 \newline
 Boutique spécialisée dans le saké. \newline
 \newline
\centerline{\includegraphics[height=90mm]{../wp-content/uploads/2015/08/P8076032-1024x768.jpg} } 
 \newline
 Un quartier avec une dizaine de temples les uns à coté des autres. \newline
 \newline
\centerline{\includegraphics[height=90mm]{../wp-content/uploads/2015/08/P8066014-1024x768.jpg} } 
 \newline
 \newline
\centerline{\includegraphics[height=90mm]{../wp-content/uploads/2015/08/P8066015-e1439361164467-768x1024.jpg} } 
 \newline
 \newline
\centerline{\includegraphics[height=90mm]{../wp-content/uploads/2015/08/P8076042-1024x768.jpg} } 
 \newline
 \newline
\centerline{\includegraphics[height=90mm]{../wp-content/uploads/2015/08/P8076053-1024x768.jpg} } 
 \newline
 \newline
\centerline{\includegraphics[height=90mm]{../wp-content/uploads/2015/08/P8076054-1024x768.jpg} } 
 \newline
 \newline
\centerline{\includegraphics[height=90mm]{../wp-content/uploads/2015/08/P8076057-1024x768.jpg} } 
 \newline
 \newline
\centerline{\includegraphics[height=90mm]{../wp-content/uploads/2015/08/P8076059-1024x768.jpg} } 
 \newline
 A coté de Takayama, un petit village de montagne reconstitué avec des maisons typiques au toit très pentu. \newline
 \newline
\centerline{\includegraphics[height=90mm]{../wp-content/uploads/2015/08/P8076066-1024x768.jpg} } 
 \newline
 \newline
\centerline{\includegraphics[height=90mm]{../wp-content/uploads/2015/08/P8076068-1024x768.jpg} } 
 \newline
 \newline
\centerline{\includegraphics[height=90mm]{../wp-content/uploads/2015/08/P8076069-1024x768.jpg} } 
 \newline
 \newline
\centerline{\includegraphics[height=90mm]{../wp-content/uploads/2015/08/P8076074-1024x768.jpg} } 
 \newline
 \newline
\centerline{\includegraphics[height=90mm]{../wp-content/uploads/2015/08/P8076072-1024x768.jpg} } 
 \newline
 Je teste le boeuf de Hida très réputé au Japon. \newline
 \newline
\centerline{\includegraphics[height=90mm]{../wp-content/uploads/2015/08/P8076077-1024x768.jpg} } 
 \newline
 Je roule ensuite vers Shirakawago, village inscrit à l'Unesco avec une centaine de maisons traditionnelles comme les précédentes. Je m'aperçois que le col pour y accéder est fermé, ce qui m'oblige à un grand détour. Finalement je renonce à Shirakawago et je descends vers la petite ville de Gujō. \newline
 \newline
\centerline{\includegraphics[height=90mm]{../wp-content/uploads/2015/08/P8086081-1024x768.jpg} } 
 \newline
 \newline
\centerline{\includegraphics[height=90mm]{../wp-content/uploads/2015/08/P8086085-1024x768.jpg} } 
 \newline
 \newline
\centerline{\includegraphics[height=90mm]{../wp-content/uploads/2015/08/P8096092-1024x768.jpg} } 
 \newline
 Chateau de Gujō perché sur une colline. \newline
 \newline
\centerline{\includegraphics[height=90mm]{../wp-content/uploads/2015/08/P8096111-1024x768.jpg} } 
 \newline
 \newline
\centerline{\includegraphics[height=90mm]{../wp-content/uploads/2015/08/P8096100-1024x768.jpg} } 
 \newline
 Belle rivière ou beaucoup de monde se baigne ou pêche. \newline
 \newline
\centerline{\includegraphics[height=90mm]{../wp-content/uploads/2015/08/P8096094-1024x768.jpg} } 
 \newline
 \newline
\centerline{\includegraphics[height=90mm]{../wp-content/uploads/2015/08/P8096113-1024x768.jpg} } 
 \newline
 Plusieurs ruelles avec des canaux. \newline
 \newline
\centerline{\includegraphics[height=90mm]{../wp-content/uploads/2015/08/P8096116-1024x768.jpg} } 
 \newline
 \newline
\centerline{\includegraphics[height=90mm]{../wp-content/uploads/2015/08/P8096096-1024x768.jpg} } 
 \newline
 Source d'eau très pure au coeur de la ville. \newline
 \newline
\centerline{\includegraphics[height=90mm]{../wp-content/uploads/2015/08/P8096114-1024x768.jpg} } 
 \newline
 C'est de Gujō que sont originaires les reproductions de plats des restaurants japonais. On peut en acheter des centaines de différentes dans cette boutique. \newline
 \newline
\centerline{\includegraphics[height=90mm]{../wp-content/uploads/2015/08/P8096115-1024x768.jpg} } 
 \newline
 Tout l'été, un festival de danse a lieu dans une rue différente chaque soir. La plupart des gens dansent et beaucoup portent le kimono, belle ambiance. \newline
 \newline
\centerline{\includegraphics[height=90mm]{../wp-content/uploads/2015/08/P8096123-1024x768.jpg} } 
 \newline

\newpage
 
